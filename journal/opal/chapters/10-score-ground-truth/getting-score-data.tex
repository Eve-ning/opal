We yield raw score data from \href{https://data.ppy.sh/}{osu! Database}.
For this project, we download

\verb+2022_04_01_performance_mania_top_10000.tar.bz2+

This contains high score data from the top 10,000 players.

\begin{itemize}
    \item \verb+osu_beatmaps.sql+ Contains map info
    \item \verb+osu_scores_mania_high.sql+ Contains score info
    \item \ldots
\end{itemize}

However, it's in \verb+.sql+, to make it easier downstream, we convert to \verb+.csv+

\subsubsection*{Conversion of SQL to CSV}

Luckily, Google Cloud (GC) has capabilities on converting it, albeit roundabout.

\begin{enumerate}
    \item Upload \verb+.sql+ files to GC Storage
    \item Migrate \verb+.sql+ files from GC Storage into GC SQL
    \item Query data using BigQuery into \verb+.csv+
\end{enumerate}

We queries as follows

\begin{figure}[H]
    \centering
    \begin{minted}{sql}
    SELECT * FROM EXTERNAL_QUERY(
        "projects/project-name/...",
        "SELECT * FROM osu_beatmaps WHERE playmode = 3;"
    );
    \end{minted}
    \caption{Beatmap Info Query}
    \label{fig:map_query}
\end{figure}

We selected \verb+playmode = 3+ to avoid machine generated beatmaps, which are not representative
of our true user-made map population.

\begin{figure}[H]
    \centering
    \begin{minted}{sql}
    SELECT * FROM EXTERNAL_QUERY(
        "projects/project-name/...",
        "SELECT * FROM osu_scores_mania_high;"
    );
    \end{minted}
    \caption{Score Info Query}
    \label{fig:score_query}
\end{figure}

We yield 11,453,737 million scores alongside 10,766 beatmaps from this dataset.


